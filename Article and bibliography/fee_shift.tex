\documentclass{article}
%\documentclass[9pt]{extarticle}
%\usepackage[margin=1.5in]{geometry}
\usepackage[
backend=bibtex]{biblatex}
\usepackage{graphicx}
\usepackage [english]{babel}
\usepackage [autostyle, english = american]{csquotes}
\MakeOuterQuote{"}
\usepackage{amsmath}
\DeclareMathOperator*{\argmax}{arg\,max}
\DeclareMathOperator*{\argmin}{arg\,min}
\usepackage{BOONDOX-cal} % a calligraphic font that includes lowercase letters, will be used with mathcal command
\usepackage{babel, blindtext}
\usepackage{caption}
\usepackage{hyperref}
\usepackage{amssymb}
\newenvironment{nohyphen}
  {\tolerance=1% Also consider setting \pretolerance
   \emergencystretch=\maxdimen%
   \hyphenpenalty=10000%
   \hbadness=10000}% \begin{nohyphen}
  {\par}% \end{nohyphen}
 
\addbibresource{fee_shift.bib}

\begin{document}

\title{Modeling Fee Shifting \\ With Computational Game Theory}
\author{Michael Abramowicz \\ \href{mailto:abramowicz@law.gwu.edu}{abramowicz@law.gwu.edu} \\ George Washington University Law School}

\maketitle

\begin{abstract}
\begin{nohyphen}
This article adapts a linear programming algorithm to identify exact perfect Bayesian Nash equilibria in litigation games represented by 1,250 extensive-form game trees of approximately 45,000 nodes each.  The game trees allow for two-sided asymmetric information, decisions of whether to file and answer, options to quit if settlement fails, variable costs, risk aversion, and costs, among other variables. Welfare variables of interest, concerning accuracy and expenditures, can be directly measured. In simulations with risk neutrality and mild risk aversion, the British rule has a minimal effect on the chance that a case will be contested at all but often leads to a lower incidence of trial. Increasing the cost of litigation in the model adversely affected social welfare but generally did not meaningfully change the advisability of fee shifting. Numerous visualizations of equilibria are offered, and over 20,000 more are included in the Supplemental Materials. 
\end{nohyphen}
\end{abstract}

\section{Introduction}
The literature analyzing the effects of fee shifting confronts a daunting analytic challenge. Settlement bargaining is a two-player asymmetric information game. The gold standard solution to such a game is a pair of common knowledge strategies that form a perfect Bayesian equilibrium. The perfection requirement, as defined by Fudenberg and Tirole (1991) \cite{fudenberg}, specializes the general Nash \cite{nash} equilibrium criterion in an imperfect information game, by insisting that at no point in the game may a player have any incentive to change the player's strategy. Each player applies Bayesian reasoning to incorporate new information, such as opponent settlement offers, into estimates of trial outcomes, and the player's probabilistic beliefs are required to be correct given such information. Complicating the challenge of crafting such equilibria is that the the underlying litigation game may include various decisions besides settlement, such as the plaintiff's decision whether to file suit, the defendant's whether to answer or default, and both parties' whether to bear the costs of trial should settlement fail. One or both parties may be risk averse, costs may be asymmetric, and so may be the quality of the players' information. A case may concern liability or damages. The loser may or may not be required to pay the winner's fees.

Incorporating anywhere near all of these considerations into a single model of settlement bargaining has proven elusive. The settlement bargaining modeler stands before a smorgasbord of potentially critical game features, but faces the admonition to choose no more than a few. The result, Daughety and Reinganum (1993) \cite{daughetyreinganum1993} observed, is a literature that "has grown in a disorganized fashion, resulting in a multitude of models involving different informational endowments and timing structures." This statement remains true nearly three decades later, with the  settlement-bargaining-modeling art making unmistakable but limited progress. The earliest models of Landes (1971) \cite{landes}, Posner (1973) \cite{posner}, and Gould (1973) \cite{gould} had ignored the challenges of Bayesian inference. Later came models, such as Bebchuk (1984) \cite{bebchuk84} and Polinsky and Rubinfeld (1998) \cite{polinskyrubinfeld}, in which one party knows the probability of liability or the amount of damages while the other party knows only the distribution, and Daughety and Reinganum (1994) \cite{daughetyreinganum1994}, in which one party has information on liability and the other party, on damages. The latest generation of scholarship, including Friedman and Wittman (2006) \cite{friedmanwittman}, Klerman, Lee, and Liu (2018) \cite{klermanleeliu}, and Dari-Mattiacci and Saraceno (2020) \cite{darimatiaccisaraceno}, models two-sided asymmetric information, in which each of the plaintiff and defendant has independent private information about the same issue, for example about the level of damages. The most recent of these even succeeds at the Herculean task of incorporating fee shifting, but we will see that even it does not escape the curse of dimensionality, adopting a number of restrictive assumptions that make it difficult to assess the generality of its conclusions.

The literature is extraordinarily clever, in both the positive and negative senses of the word. It takes advantage of mathematical assumptions to make otherwise intractable problems tractable. We can expect further progress from relaxing different assumptions, but the goal of developing a single mathematical model that allows exploration of different values of a large number of variables may be unattainable. The literature develops critical intuitions about settlement bargaining, including how changing fee shifting rules may augment or diminish the effectiveness of the litigation system, and review articles, like Katz and Sanchirico (2012) \cite{katzsanchirico}, informally integrate various models' conclusions about how fee shifting might affect trial rates or settlement rates with different structures to the litigation game. Even with such reviews, however, it is difficult to generalize about the wisdom of fee shifting, because of the interactivity between trial and settlement rates. If, for example, increased ease of settlement leads to plaintiffs' bringing and defendants' defending more cases, total litigation expenditures could even rise. Moreover, overall litigation accuracy depends on what the correct answer would be in a hypothetical perfect litigation system and how this compares with the net receipts of plaintiffs or outlays of defendants, taking into account damages awarded, settlements agreed to, and litigation costs. Even Dari-Mattiacci and Saraceno (2020), which focuses on accuracy, offers a measure that ignores that a successful meritorious plaintiff has not achieved a meaningfully accurate result if most of the winnings are spent on litigation. This observation is not intended as a criticism. Mathematical models of the settlement process are difficult to construct, and calculating the derivative of accuracy, itself a complicated function of many other variables, with respect to variables like costs or fee-shifting intensity is more difficult still, at least without other serious simplifications.

Scholars have studied settlement bargaining with other methodologies, but these have their own limitations. Empirical analyses are limited to studying the rare cases in which a change in fee-shifting rules occurs, as in the examination by Hughes and Snyder (1995) \cite{hughessnyder} of a briefly-lived policy experiment in Florida. Laboratory experiments offer another approach, with contributions by Coursey and Stanley (1988) \cite{courseystanley}, Inglis et al. (2005) \cite{inglisetal},  Rowe and Vidmar (1988) \cite{rowevidmar}, and Main and Park (2000) \cite{mainpark}. It is not clear, however, whether modest stakes produce results similar to those of real litigation. A final methodology in the literature is computer-based simulation. Priest and Klein (1984) \cite{priestklein}, Katz (1987) \cite{katz}, Hause (1989) \cite{hause}, and Hylton (1993) \cite{hylton}, have all used computation, either independently or as complements to formal models. Although these articles all include innovations building on the Landes-Gould-Posner model, they share a significant limitation: Unlike the math models, the simulations do not seek perfect Nash equilibria.

It is, however, possible to harness computational power in the quest for perfect Bayesian equilibria, by turning to computational game theory. The settlement bargaining literature has acknowledged the importance of game theoretic concepts, but it generally does not explicitly build game trees. Acknowledging that litigation can be viewed as "a particular extensive-form bargaining game," Spier (1994, pp. 202-03) \cite{spier} sensibly worries that the results would be sensitive to issues such as "the structure of the asymmetric information." This concern suggests that solvable game theoretic models are insufficiently rich to encapsulate critical aspects of the litigation game. Computational game theory, however, allows for the identification of equilibria in games that could not practically be solved by hand. A subliterature focuses directly on the solution of two-player (and sometimes $n$-player) general sum games. A litigation game between plaintiff and defendant is general sum, which can be more difficult than a zero-sum game to solve, because the players may transfer wealth not only from defendant to plaintiff, but also from both litigants to lawyers. A useful review of algorithms that can help solve such games is von Stengel (2002) \cite{vonstengel2002}.

A publicly available open source software package known as Gambit by McKelvey et al. (2016) \cite{mckelvey} features a number of these algorithms. This article, however, applies an algorithm not included in Gambit, specifically an algorithm described in an article in \textit{Econometrica}: von Stengel, van den Elzen, and Talman (2002) \cite{vonstengelvandenelzentalman}. This algorithm, described further below, is guaranteed to produce perfect Bayesian Nash equilibria in a finite game in which the players have perfect recall. Sometimes, these equilibria are pure, with players acting deterministically conditional on the information that they possess, but at other times, they are mixed, with the players randomly choosing at certain moments of the game between or among equally good strategies, each with some nonzero probability. Mixed strategies need not reflect explicit randomization by players; they may be understood as describing balanced populations of litigants who take different approaches, none better than others given opponents' strategies, for reasons exogenous to the model. The authors test their algorithm on games of up to 1,023 nodes. This article pushes the computational limits of the algorithm, applying it to each of a large number of games with up to 45,211 nodes, after simplification of a game tree much larger than that into a mathematically equivalent tree. By separately identifying equilibria corresponding to different information and game structures, we can assess the conditions in which changing fee-shifting rules may increase or decrease the effectiveness of the litigation system, as manifested in accuracy or expenditures, assuming the litigation game is played in equilibrium by rational actors. This approach thus enables modeling of a richer and more diverse litigation environment than any single prior approach.

Indeed, the modeling options are so great that the number of results produced dwarfs what can be comprehensively explored in a single article. The variables include the fee shifting rule, the extent of fee shifting, costs, risk preferences, the endogenous distribution of litigation quality, the quality of the legal system, the strength of the informational signals parties receive, when costs are borne across time, whether parties are permitted to quit litigation before trial, and whether the issue concerns liability or damages. Even considering only the values of these variables used in the simulations would leave 168,000 permutations, each requiring a separate simulation for each equilibrium to be derived. To make the project more manageable, the variables were divided into more and less important, with a wide range of values of the more important variables permuted with each of the less important variables considered separately and thus not permuted against one another. This resulted in 1,250 different permutations, with a separate simulation required to generate a single equilibrium for each. 

Though a more manageable number, even that is more than can be described here. As a result, supplemental materials are available in the form of an online repository, consisting of over 20,000 diagrams, some producing results for a single equilibrium and others illustrating how changing variables may affect outcomes. All diagrams are included as Latex code, and those corresponding to the full game tree are also compiled into .pdf files. The Supplemental Materials repository also includes spreadsheets for each of the simulations with two different game tree sizes, files containing the equilibria derived, and log files. The Supplemental Materials are available at http://github.com/mbabramo/Fee-shifting-article/tree/Shorter. In addition, the open source code used to develop the article, partly adapted from source code published at https://github.com/stengel/ecta2002, is available at https://github.com/mbabramo/ACESim4/tree/feeshifting. The ReadMe.txt file in the root directory contains instructions for replicating all the simulations, performing simulations on permutations of variables not explored here, and also for producing the diagrams.  It also identifies the code that verifies that all the equilibria found by the algorithm are indeed exact perfect Bayesian equilibria, ruling out the possibility of an error in implementing the algorithm.

Part 1 describes both the game trees to which the algorithm will be applied and the algorithm itself, and it reports statistics on how the size of the game tree affects the running time of the algorithm. Part 2 provides the central results. It begins by scrutinizing the results of two simulations, one in which the American rule applies and one in which the British rule applies, where all other variables are set to baseline values (admittedly values not calibrated to any particular real-world values). This provides some tentative findings, but explores a tiny proportion of the variable space. It then illustrates how a range of values for fee-shifting multipliers and the cost of litigation affect case dispositions (including filing and answer decisions, settlements, and decisions to quit after settlement failure), which in turn affect litigation accuracy and total expenditure levels. Then, the article deviates from baseline parameter values to explore the implications of risk aversion, while continuing to vary cost and fee-shifting levels. We leave discussion of other variables to future work. Part 3 provides robustness tests of the principal results by considering a much larger number of equilibria identified for a smaller version of the game tree. This permits assessment both of whether the results are sensitive to the granularity of the game tree and whether it matters whether the players play an individual equilibrium, a correlated equilibrium or an average equilibrium. Part 4 provides some conclusions and explains why they differ from earlier findings in the literature.

\section{Litigation as an Extensive Form Game}

This section frames the fee-shifting problem in explicitly game theoretic terms by describing the extensive-form game tree, and it then offers an overview of the algorithm the article uses to solve it. The last subsection reports on the performance of the algorithm in practice when applied to the game tree. 

\subsection{The Game Tree} \label{gametree}



To use a computational approach that finds equilibria from extensive form game trees, the model must encompass a discrete number of values representing the strength of the plaintiff's case, the signal that each party receives about the strength of the case, and the settlement values that each party can offer the other. In most of the simulations, we will allow for $n_{LS}=10$ quality levels on the issue of liability, $n_{LS}^P=n_{LS}^D=10$ signals of liability strength for each of the plaintiff and defendant on the issue of liability (damages will be considered separately later), and  $n_{\mathcal{o}}=10$ offers for each party. 

A model dependent on discrete variables is admittedly less general than a model based on continuous quantities, and a virtue of the mathematical literature on settlement bargaining is that some variables can be modeled continuously. In Bebchuk (1984), for example, the defendant knows the probability that the court will find liability, which can be any value from a mutually known probability density function. But not all variables in mathematical settlement models are continuous. In their review of the settlement bargaining literature, Daughety and Reinganum (2012) \cite{daughetyreinganum2012} include both two-types and continuous types models. Even models that offer continuous types in one respect feature only two types in another respect. For example, Klerman, Lee, and Liu (2018) \cite{klermanleeliu} allow each party to obtain a continuous signal of the quality of litigation, but the quality itself can take on only one of two values. Moreover, the one-sided asymmetric information models of litigation can be considered one-type models, at least with respect to the party without information.  Thus, the conventions of the existing literature do not mandate continuous variables. The 10 quality levels and signals of liability strength for each party allow a considerably greater degree of granularity than two-types models. 

To describe the litigation game, however, it is easier to focus on a much smaller game tree, where $n_{LS}=n_{LS}^P=n_{LS}^D=n_{\mathcal{o}}=2$. This smaller game tree, including probabilities corresponding to a perfect equilibrium of the game (produced to three decimal places), is presented in Figure \ref{fig:gametree2x2x2}. This is presented in full at the expense of legibility to illustrate that even with these low parameter values, the game tree is fairly large. (The full figure is available in the Supplemental Materials.) Thus, we will simplify further, focusing on particular parts of the game tree.
\begin{figure}[h!]
\centering
\includegraphics[width=10cm, height=10cm, trim={0in 0in 0in 0in}, clip]{../Figures/game tree 2x2x2.pdf}
\caption{A bird's eye view of a small game tree}
\label{fig:gametree2x2x2}
\end{figure}

Figure \ref{fig:gametree2x2x2beginning} illustrates the beginning of the litigation game. The first decision in the game belongs neither to the plaintiff nor the default, but to Chance, abbreviated C, a nonstrategic player. This common device, known as the Harsanyi transformation after Harsanyi (1967) \cite{harsanyi}, effectively creates a random number generator, transforming a game of perfect but incomplete information (where each player knows all moves that have been made but does not know payoffs) into one of complete but imperfect information (where a player may not know of some moves, such as Chance decisions revealed only to another player, but knows the complete game structure). Chance's initial decision in this game is whether the case is one in which the defendant is truly liable or one in which the defendant is not truly liable. Chance chooses true liability with a probability $p_{TL}$, here $\frac{1}{2}$. 

Explicit modeling of true liability will allow us ultimately to measure the accuracy of outcomes explicitly. A less ambitious approach would be to define accuracy solely based on whether settlements mimic trial outcomes, but that would ignore the imperfections of judgments themselves. A more ambitious approach, to be pursued in future work, would be to model the emergence of disputes endogenously in some particular legal context, as in Hylton \cite{hylton}, in which a potential tort defendant initially makes a decision about taking care. The approach here, however, is more general, and by varying Chance's probabilities, we will be able to change the distribution of true liability. 

\begin{figure}[h!]
\centering
\includegraphics[scale=0.25, trim={0in 0in 0in 0in}, clip]{../Figures/game tree 2x2x2 beginning.pdf}
\caption{The beginning of the small game tree}
\label{fig:gametree2x2x2beginning}
\end{figure}

Chance then immediately faces a second decision, what the strength of the plaintiff's case on liability should be. Using the baseline parameters (to be described below), when the defendant is not truly liable, there is an 85.1\% chance that the liability strength will be low, and when the defendant is truly liable, there is an 85.1\% chance that the liability strength will be high. This approach recognizes that although there will generally be a correlation between true liability and the strength of a plaintiff's case, there may be cases in which a plaintiff has a weak case despite true liability, or a strong case despite absence of true liability. By separating true liability from the strength of the liability case, we can assess how the strength of the legal system, defined loosely here as the correlation of true liability with case strength, affects the case for fee-shifting. 

Each case has a liability strength that is a noisy signal of the true liability value. Then, each party in turn receives a noisy signal of the liability strength,  As illustrated in Figure \ref{fig:gametree2x2x2beginning}, if the liability strength is low, then the plaintiff is more likely to receive a low signal than a high signal, and vice versa if the liability strength is high. The defendant's signal is determined independently of the plaintiff's, but the two are correlated, because each is determined in part by the true liability level. Although the game tree imposes an ordering, making it appear that the Chance decision determining the plaintiff's signal occurs before the Chance decision determining the defendant's signal, that is irrelevant and has no effect on equilibrium determination, because the defendant does not learn of the plaintiff's signal and vice versa.

Figure \ref{fig:liabilitysignalsdefault} provides a simple visualization of the probabilities of different liability strengths given the true liability status, and of the probabilities that a player will receive different liability signals. These probability values correspond to the baseline values that this article will use in its simulations. The degree of noise is admittedly somewhat arbitrary. A future project would be to calibrate signals based on empirical data for some particular category of cases, but the observation of Gelbach (2018) \cite{gelbach} that many very different models may share a common reduced form suggests that this will be challenging. An informal assessment based on Figure \ref{fig:liabilitysignalsdefault} might conclude that the assumed noise values chosen correspond to a legal system that is pretty good. It is relatively rare, for example, for a truly liable case to have a below-average liability strength signal, or for a party to receive a signal indicating likely liability when the actual liability strength is below average. This point highlights that the highly strategic play and relatively low settlement rates that we will observe with risk neutrality are not the result of weak party information. We will, however, also consider the effects of both stronger and weaker information, as well as of asymmetric information.

\begin{figure}[h!]
\centering
\includegraphics[scale=0.4, trim={0in 0in 0in 0in}, clip]{../Figures/liability signals default.pdf}
\caption{Liability strength given true liability and party's signal given liability strength}
\label{fig:liabilitysignalsdefault}
\end{figure}


A litigation game equilibrium can be calculated for any distribution of signals given true values, so a modeler could simply exogenously specify the conditional probabilities illustrated in Figure \ref{fig:liabilitysignalsdefault}. It is useful, however, to encapsulate the degree of noise into simple parameters. The probabilities of liability strengths in Figure \ref{fig:liabilitysignalsdefault} depend on a parameter $\sigma_{LS}$, which represents the standard deviation of a normal distribution from which noise values are drawn. A noise value is summed with a value taken from the true liability Bernouilli distribution, so the greater $\sigma_{LS}$, the more attenuated the connection between whether a defendant is truly liable and what the liability strength is. The signals of liability strength that the plaintiff and defendant receive depend on the similar parameters $\sigma_{LS}^P$ and $\sigma_{LS}^D$. In the baseline simulations represented in Figure \ref{fig:liabilitysignalsdefault}, $\sigma_{LS}=0.35$ and $\sigma_{LS}^P=\sigma_{LS}^D=0.20$.

More formally, the unit interval is divided into $n_{LS}$ equal segments, corresponding to the $n_{LS}$ values. The probability of each liability strength level is equal to the relative probability that the sum falls into the corresponding segment, ignoring sums less than 0 or greater than 1. That is, let $TL=\{0,1\}$ represent the possible true liability values, corresponding to not truly liable and truly liable, respectively; let $LS_i=\frac{i - \frac{1}{2} }{n_{LS} }$ for $\forall i \in \{1,2,...,n_{LQ}\}$ represent the set of liability strength values; and let $N_{LS,k}=\langle\sigma_{LS}\Phi^{-1}(\frac{i}{k}) \rangle_{i=1}^{k-1}$ represent a sequence of $k-1$ noise values drawn from the inverse normal  distribution with standard deviation $\sigma_{LS}$. Then, $\forall t\in TL$, let 
\begin{equation} 
p_{LS_i|t} = \lim_{k\to\infty} \frac{\lvert\{t+\nu |\nu \in N_{LS,k}, LS_i - \frac{1}{2n_{LS} }<t+\nu \leq LS_i + \frac{1}{2n_{LS} }\}\rvert}{\lvert\{t+\nu |\nu \in N_{LS,k}, 0<t+\nu \leq 1 \}\rvert}
\end{equation}
represent the probability that the signal with index $i$ is received given a truly liable value of $t$.

For player $j$, the probability of receiving a particular signal of liability strength can be calculated as
\begin{equation} 
p_{LS_i^j|LS_i} = \lim_{k\to\infty} \frac{\lvert\{LS_i+\nu |\nu \in N_{LS,k}, LS_i^j - \frac{1}{2n_{LS} }<LS_i+\nu \leq LS_i^j + \frac{1}{2n_{LS} }\}\rvert}{\lvert\{LS_i+\nu |\nu \in N_{LS,k}, 0<t+\nu \leq 1 \}\rvert}
\end{equation}

\noindent The volume of the flows from the "not truly liable" and "truly liable" nodes to each of the liability strength nodes correspond to the $p_{LS_i|t}$ probability values, and the volume of the flows from the liability strength nodes to the signal nodes correspond to the $p_{LS_i^j|LS_i}$ values. 

Returning to our smaller game tree with just two liability strengths and two signals for each party in Figure \ref{fig:gametree2x2x2beginning}, each path leads to a decision to be made by the plaintiff. An information set number is indicated next to the "P" indicating plaintiff. The information sets labeled "P0" correspond to the cases in which the plaintiff receives the low signal of liability strength, while the information sets labeled "P12" correspond to the cases in which the plaintiff receives the high signal. (The missing information set numbers between 0 and 12 appear in the endgame, to be presented momentarily.) This signifies that the plaintiff cannot distinguish among all of the cases within each information set. That is, the plaintiff does not know whether the case is one that is truly liable or not truly liable, and the plaintiff does not know whether the defendant receives a low signal or a high signal. The plaintiff's equilibrium strategy must be optimal given the constraint that the probabilities that the plaintiff assigns to its next move must be the same, regardless of the true liability and of the defendant's signal, and thus conditional only on the plaintiff's signal. 

Figure \ref{fig:gametree2x2x2end} illustrates the decision to be made by the plaintiff at the "P0" information set, corresponding to the very top line in Figure \ref{fig:gametree2x2x2beginning}. The decision is whether to file suit. In this equilibrium, we can see that the plaintiff always files suit in this information set. If, however, the plaintiff did not file, then the game would end, with each player receiving a utility of 10. The value 10 is an arbitrary value used to represent each party's initial wealth; because this particular game tree represents risk neutral utilities, the initial wealth is irrelevant, and the equilibrium would be exactly the same for any linear transformation of one or both players' utility values. If the plaintiff does file suit, then the plaintiff will incur a cost $c_{file}$, which is equal to 0.15 in this version of the game tree. The defendant then faces a decision whether to answer, which also causes the defendant to incur a cost $c_{answer}=0.15$. If the defendant does not answer, then the plaintiff wins, and the defendant pays damages that in all simulations in this article are normalized to 1.0. Thus, plaintiff receives a utility of 10.85, and the defendant, 9.0.

\begin{figure}[h!]
\centering
\includegraphics[scale=0.25, trim={0in 0in 0in 0in}, clip]{../Figures/game tree 2x2x2 end.pdf}
\caption{The end of the game tree}
\label{fig:gametree2x2x2end}
\end{figure}

Assuming that the plaintiff files and the defendant answers, then the plaintiff faces a decision whether the plaintiff will abandon the litigation rather than incur trial costs if settlement fails, and the defendant similarly faces a decision whether to default in the event settlement fails. Neither party's decision has any immediate effect, and neither party finds out about the other party's decision. The placement of these decisions before settlement rather than after is admittedly counterintuitive. But it greatly speeds up the algorithm for calculating equilibria, relative to a model in which each player takes into account both its own settlement offer and its opponent's in deciding whether to quit. With the latter approach, there would be one information set for each player for each triple of a liability signal, a settlement offer, and an opponent's settlement offer. By deciding abandonment or defaulting in advance, we greatly reduce the number of information sets in the game, with just one information set concerning quitting for each liability signal for a player. Although it might be interesting to study the strategic interaction in a model in which a party's offer may affect the opponent's decision whether to quit, that is beyond our scope here. Including decisions whether a party will abandon or quit litigation continues to serve its primary role of affecting bargaining indirectly, by forcing each party to consider in making a settlement offer whether its opponent has a credible threat to take a case to trial. Commentators including Huang (2004) \cite{huang} and Hubbard (2016) \cite{hubbard} have stressed that even if quitting just before trial is rare, the possibility of such quitting may discipline decisions on whether to file or answer, as well as settlement offers.

Whatever the future plans the parties make about whether to abandon or default, each party makes a settlement offer to the other party. In the diagram, it appears that the plaintiff makes an offer before the defendant, but this timing is again irrelevant. Note that the defendant faces the same information set regardless of whether the plaintiff makes a low offer or a high offer. That is, the defendant does not learn the plaintiff's decision before it makes its own, so the decisions are effectively simultaneous. 

The bargaining protocol used is that introduced by Chatterjee and Samuelson (1983) \cite{chatterjeesamuelson}. If the plaintiff's offer exceeds the defendant's, the case settles at the midpoint; otherwise, bargaining has failed, and, unless a party has decided to quit, the case goes to trial. As Friedman and Wittman \cite{friedmanwittman} explain, this choice is justified not on the ground that the protocol is commonly used (it is not), but on the ground that it provides a useful reduced form of a more complicated bargaining process. While future work might employ some other bargaining protocol, such as back-and-forth offers between the parties, that would exponentially increase the number of information sets and the amount of time needed for the algorithm to run. Chaterjee-Samuelson bargaining avoids the simplification in the Landes-Gould-Posner bargaining models of assuming that a case will settle whenever that is in the mutual interest of the parties, given their expectations. With Chaterjee-Samuelson bargaining, parties have some incentive to shade their offers in their own favor in the hope of capturing more of the settlement surplus, though they must balance this objective with the risk of bargaining failure. The approach thus allows for an important dimension of strategic bargaining, while avoiding the simplification adopted by models that arbitrarily choose a player to make a take-it-or-leave-it offer to the other player.

Finally, at the far right end of Figure \ref{fig:gametree2x2x2end} are the utilities received by the players in cases that settle, as well as the further game play in cases that do not settle. If a case does not settle, but one player previously decided to quit in the event of settlement failure, then the game immediately ends without trial. If both players decided to quit, which in practice is likely to occur only when the cost of trial is much greater than the level in Figure \ref{fig:gametree2x2x2end}, then a chance decision assigns each a 50\% chance of quitting, as demonstrated near the top right of the diagram. In the more important case where neither party quits after settlement failure, trial occurs. Each party is charged a cost of $c_{trial}=0.15$, and  the court must determine whether to impose liability. The court makes its decision based on a noisy assessment of liability strength, must as the players do; thus, the modeling allows not only for liability strengths that differ from truth but also for different levels of judicial idiosyncracy. The parameter determining the noisiness of the signal that the court faces is the same as that for the players, $\sigma_{LS}^C=0.2$. The court, however, observes one of only two liability signal levels. If it observes the higher level, the court imposes liability. In this game tree, no fee shifting occurs, so liability results in a transfer from the defendant to the plaintiff of 1.0.  Note that the court is making an inference based on its signal, but this is not a fully Bayesian inference. The court does not take into account the initial distribution of disputes, and it does not refine its inferences based on game play, for example by determining whether a failure of settlement is more likely when the defendant is or is not truly liable. A project for future work, more in the spirit of the mechanism design literature such as Spier (1994) \cite{spier}, would be to make the court an additional player in the game that can determine the level of damages and seeks to optimize social welfare.

The game tree can be simplified when calculating equilibria. The Chance decisions determining true liability and the liability strength do not directly result in any information being added to the plaintiff's or defendant's information set. The only information from a Chance decision that matters to the litigants is the signal of liability strength. We can thus skip the first two Chance decisions and jump directly to the stage where the plaintiff receives a signal, so long as the probabilities of each signal are the same for the plaintiff and so long as the probabilities that the defendant receives each signal conditional on any signal received by the plaintiff remain the same. Figure \ref{fig:gametree2x2x2beginning_reduced} shows a reduced form version of the beginning of the game tree. The probability that the defendant receives the same signal as the plaintiff, 79.2\%, can be computed in the larger game tree by weighting the probability of different branches. Using this reduced form will produce the same equilibria as will result from following Figure \ref{fig:gametree2x2x2beginning}. 

\begin{figure}[h!]
\centering
\includegraphics[scale=0.25, trim={0in 0in 0in 0in}, clip]{../Figures/game tree 2x2x2 beginning simplified.pdf}
\caption{A reduced-form version of the beginning of the game tree}
\label{fig:gametree2x2x2beginning_reduced}
\end{figure}

The end of the game can also be simplified. The court's eventual decision on liability affects the players, but a model in which a player receives the expected value of the court's decision with certainty produces identical equilibria. Meanwhile, in the rare circumstances when both parties decide to quit, resolving the case at an average of the utilities that would obtain when one quit produces the same result as flipping a coin to determine whether the plaintiff wins or loses. In both cases, this works regardless of risk aversion level, so long as the final outcomes are utilities, rather than risk levels. Figure \ref{fig:gametree2x2x2end_reduced} illustrates the reduced-form version of the end of the game tree in the very small game.

\begin{figure}[h!]
\centering
\includegraphics[scale=0.25, trim={0in 0in 0in 0in}, clip]{../Figures/game tree 2x2x2 end simplified.pdf}
\caption{A reduced-form version of end of the game tree}
\label{fig:gametree2x2x2end_reduced}
\end{figure}

The full game tree will still be needed to calculate variables such as party win rates at trial and accuracy of the system of litigation. Thus, after the equilibria are calculated using the simplified version of the tree, the full tree can be explored to determine the exact probability of every outcome. None of the results in this article depends on Monte Carlo sampling, and so no error bars are needed, given the exact equilibrium identified by the algorithm.

\subsection{The von Stengel, van den Elzen, and Talman Algorithm}

In principle, any two-player extensive form game can be presented in strategic form, with the parties' payoffs embedded in a matrix. The row player may use a pure strategy and select a single row from the matrix or a mixed strategy, a probability distribution over the matrix rows. The column player analogously chooses a column or a probability distribution over matrix columns. Nash (1951) famously proved that every strategic game involving a finite number of players has at least one equilibrium, now called a Nash equilibrium, in which neither player could increase the player's payoff by switching to a different strategy given the opponent's strategy. Moreover, given a game in strategic form, any equilibrium in pure strategies can be solved relatively easily by the algorithm of iterative elimination of dominated strategies. If a cell of the matrix exists where the column player's utility is greater than in any other cell in its row and the row player's utility is greater than in any other cell in its column, then that cell represents a Nash equilibrium. 

This approach is insufficient to find mixed strategy equilibria. The problem of finding Nash equilibria, however, can be converted into a linear programming problem. The reader with a strong interest in the algorithm should read von Stengel (2002) for a comprehensive overview, including proofs that it produces equilibria. The reader interested solely in legal ramifications may skip this section altogether. A virtue of using computational game theory to identify equilibria is that one may choose to treat algorithms as black boxes, particularly because all equilibria found in this article were computationally verified. The reader with an intermediate level of interest in the algorithm will find a brief introduction to it in the Supplemental Materials.

The algorithm is not without its limitations. First, the algorithm is not guaranteed to produce all Nash equilibria. One may attempt to obtain multiple equilibria by choosing different initializations of the information sets. In most of this article, we simply obtain the one equilibrium corresponding to the evenly mixed strategies, but as a robustness check, we will seek out multiple equilibria in a smaller version of the game. 

Second, to ensure that an equilibrium is reached, the algorithm must be performed using exact arithmetic, rather than floating point arithmetic. That is, the algorithm is not guaranteed to succeed if rounding occurs during the pivoting phase, even given the generous number of digits allowed by modern 64-bit computer architectures. A consequence of using exact arithmetic is that every utility value and chance probability must be represented as a positive rational number. Accordingly, in each simulation, utility values were scaled to integers ranging from 1 for the lowest possible utility for a player to 100,000 for the highest. The loss in precision from rounding off some utility values is immaterial, and because rounding occurs before the pivoting, it allows the algorithm to proceed. The use of exact arithmetic is cumbersome, however, requiring complications involving fractions of relatively large numbers. For example, in the baseline simulation to be reported in Figure \ref{fig:offers_american}, the defendant in equilibrium takes one particular action with an exact probability in canonical form of $\frac{7,686,146}{13,380,759}$, and the calculations leading to determination of the equilibria often involve fractions with many more digits. (For legibility, we will present this and other results using decimal approximations.) The source code accompanying the present article allows the algorithm to be performed using either exact arithmetic or floating point arithmetic. Floating point arithmetic is much faster and sometimes yields an equilibrium, but the procedure is hit or miss.

Third, in the worst case, the algorithm is exponential in the size of the strategic form game matrix. Even with a relatively small matrix, an exponential algorithm may be infeasible. Because the game tree contains every combination of plaintiff signal, defendant signal, plaintiff offer, and defendant offer, the number of nodes is $O(N^4)$. The algorithm, however, did prove able to find equilibria in each simulation tested, completing after 448 processor-core hours. In some cases, the algorithm took far longer to complete than in other cases. For example, one simulation, in which both parties had high risk aversion and there was a moderate amount of fee shifting, concluded after 6,263 pivoting steps and took nearly 8 hours to complete on a single processor core. 

\section{Results} \label{results}

In addition to furnishing a much more complex game tree than theorists have been able to achieve with pure math models, the computational game theory approach applied here enables many changes to the litigation game to be made with no increase in algorithmic complexity. Any change that affects only the players' utilities leaves the game tree the same size, and so the time needed for modeling needs to be multiplied only by a constant factor representing the number of different permutations of parameter values to be tested. Some considerations that are difficult to model mathematically are trivial computationally. Risk aversion, for example, is ordinarily complex to model, and different forms of utility function may require separate proofs, as in Park and Lee (2019). The approach here easily accommodates changes in the degree or form of risk aversion. Similarly, changes to the costs the parties bear and the quality of their information add no complications. Different fee-shifting rules and intensity of fee shifting, including requirements that the loser pay a multiple of legal fees, can easily be integrated in the model. 

This section thus considers a wide range of potential variations of its settlement bargaining model. The analysis begins with the baseline parameter values defined in the game illustrated in Figure \ref{fig:gametree2x2x2} (but with $n_{LS}=n_{LS}^P=n_{LS}^D=n_{\mathcal{o}}=10$), while varying costs and fee shifting multipliers. The parameter values are admittedly not yet calibrated to actual data, but real world parameter values are likely to vary from place to place and case category to case category in any event. Probably the most important parameter, other than the fee-shifting rule, that may vary across suits is the cost relative to the stakes. The litigation may play out very differently in a small claims case, in which total costs to litigants including lawyers and their own time will generally be high as a percentage of the stakes, than in a multi-billion dollar suit in which even the priciest lawyers will charge only a small fraction of what is at issue. Thus, even when varying other parameters, simulations considering various permutations of costs and the fee-shifting multiplier are included. Space limitations prevent us from reporting all findings here, but spreadsheets and diagrams for each simulation are available in the Supplemental Materials, including over 20,000 diagrams in PDF or Latex form, and simple changes in the source code can be used to test combinations of parameters not already simulated.

\subsection{Baseline comparison} \label{baseline}

Before varying costs, we will scrutinize the results of two simulations, the simulation corresponding to the baseline parameter values and the identical simulation except applying the British rule, so that if a case goes to trial, the loser pays the winner's costs. The model does not differentiate whether these costs are in the form of attorney fees, court costs, or litigants' time. Figures \ref{fig:fileans_american} and \ref{fig:fileans_british} illustrate the plaintiff's decision whether to file suit and the defendant's decision whether to answer (conditional on filing) in the identified equilibria. Although models like Shavell (1982) \cite{shavell} suggest that the British rule will reduce the likelihood that a plaintiff with a low likelihood of winning will file suit, no such effect is visible in these figures, as the plaintiff's decisions are identical. The defendant's decisions are almost identical too, except that when the defendant receives a signal of 0.65, the defendant has a 4\% chance of answering under the British rule, in contrast to the American rule, where the defendant never answers. 

The unexpected stasis helps illustrate the power of computed equilibria to identify dynamics not apparent in simpler models. This result is a consequence of multiple competing pressures that balance one another, pressures that would not be evident in a game that did not feature file/answer decisions, settlement offers, and decisions whether to give up after the failure of settlement. Consider a plaintiff who receives a signal of 0.35 and is thus near the margin of deciding whether to file suit. The British rule does indeed provide further impetus in the direction of not filing suit. On the other side of the liability signals spectrum, the defendant faces similar pressure not to answer. Fee shifting, however, also gives this plaintiff a greater incentive to drop the suit rather than face trial costs if settlement fails. That change in turn makes it more attractive for the defendant to fight, toughening the defendant's settlement negotiation stance. These changes can lead to further hydraulic pressures that just happen in this example to approximately cancel out. 

\begin{figure}[h!]
\centering
\includegraphics[scale=0.50, trim={0in 0in 0in 0in}, clip]{../Figures/File and answer decisions (American).pdf}
\caption{File and answer decisions (American rule)}
\label{fig:fileans_american}
\end{figure}

\begin{figure}[h!]
\centering
\includegraphics[scale=0.50, trim={0in 0in 0in 0in}, clip]{../Figures/File and answer decisions (British).pdf}
\caption{File and answer decisions (British rule)}
\label{fig:fileans_british}
\end{figure}

It might seem that this is a "just so" story and that one could construct many narratives to explain how changing a parameter leads to some change (or in this case, no change) in some decisions by the parties. Indeed, the equilibrium dynamics are sufficiently complicated that we should resist the temptation of overinterpretation. But  we can directly observe at least the immediate pressures on the file and answer decisions that result from changing from the American to the British rule. Specifically, we can change the players' utility payoffs to those that obtain with fee shifting, but retain the strategies that they play under the American rule equilibrium. Then, we can examine the utilities that each player can expect to receive at each information set. Because an information set may correspond to multiple nodes of the game tree, these utility values are the expected utilities from choosing each action at a node, weighted by the probability of reaching that node in the game tree. At equilibrium, the actions with positive probability must all have the same, maximal expected utility (otherwise, we would not be at perfect equilibrium), but changing the payoffs can alter this. We can thus identify actions that the player would choose more often, assuming that all other aspects of both parties' strategies remained fixed. 

This analysis (contained in the Supplemental Materials under the Supplemental information folder) demonstrates, for example, that, holding all else equal, the plaintiff's nonsuit rate would increase by 6.4 percentage points, and the defendant's default rate by 4.4 percentage points, if changing to the British rule while holding the other party's strategy constant. It also demonstrates how changing these values in turn would lead to the other effects described in the previous paragraph. The technique thus provides some intuition for why switching to the British rule might not lead the parties to contest litigation less often. The words "might not" highlight not that we should be unsure about whether the equilibria partially depicted in Figures \ref{fig:fileans_american} and \ref{fig:fileans_british} really exist \textemdash they do, and the source code confirmed that they are in fact perfect Nash equilibria, though of course we cannot be sure that human beings will play according to the equilibria \textemdash but that we must be attentive to the possibility that different results might obtain with other parameter values. 

We will return to exploring other parameter values, but first let us consider the offers that the parties make, conditional on the signals that they received. This is illustrated in Figure \ref{fig:offers_american} for the American rule, and Figure \ref{fig:offers_british}, for the British. Some columns feature no offers; these are the same columns for which the parties did not contest litigation. What is striking about the offers that the parties make is that they are stingy. When the plaintiff receives a signal of 0.35, the plaintiff insists on a settlement of no less than 0.75. (Recall that under Chaterjee-Samuelson bargaining, the case will settle only if the defendant's simultaneous offer is greater than or equal to the plaintiff's.) The defendant, meanwhile, when receiving a signal of 0.55 will offer no more than a nuisance settlement of 0.15. And the parties are stingier still when moving toward the extremes, at which each party insists on the stingiest possible settlement. One might have the intuition that settlement offers would rise roughly linearly with signals and that close cases would settle for middle-range values. After all, the parties do have a monetary interest in avoiding trial. 

\begin{figure}[h!]
\centering
\includegraphics[scale=0.50, trim={0in 0in 0in 0in}, clip]{../Figures/Offers heatmap (American).pdf}
\caption{Offer decisions (American rule)}
\label{fig:offers_american}
\end{figure}

\begin{figure}[h!]
\centering
\includegraphics[scale=0.50, trim={0in 0in 0in 0in}, clip]{../Figures/Offers heatmap (British).pdf}
\caption{Offer decisions (British rule)}
\label{fig:offers_british}
\end{figure}

That strategic bargaining might occur is not surprising. All of the Bayesian models of settlement that use Chaterjee-Samuelson bargaining (including Friedman and Wittman (2007) \cite{friedmanwittman}) recognize that each party faces a trade-off: Greater generosity increases the chance of avoiding trial, but greater stinginess increases the portion of the bargaining surplus that the offeror will receive if a settlement does occur. But strategic bargaining plays a considerably larger role than one might expect. Recall from Figure \ref{fig:liabilitysignalsdefault} that the amount of noise obscuring each party's evaluation of the merits does not seem all that large, and yet considerable strategic behavior manifests. The pattern of offers observed here is hardly invariant, but settlement behavior of this sort occurs under a wide range of parameter values, especially when the parties are risk neutral. Real-world bargaining includes many features not accounted for in this model, but this result tentatively suggests that litigants might rationally often fail to reach settlements even when their actual assessments of the litigation value are not far apart. At least, these results should make us cautious about accepting at face value the results of Landes-Posner-Gould models that assume settlement occurs whenever settlement would produce social surplus.

The considerable extent of strategic bargaining does have an upside for this project. If the analysis revealed that parties' settlement offers were often very close to each other, then one might worry that the discreteness of action choices in this game theoretical model might be affecting results. Suppose, for example, that a change in some parameter would lead in a continuous model to slight changes in the parties' settlement functions that nonetheless had large effects on the settlement rate. It would be hard for a discrete model, in which each party must choose among only ten offer levels, to capture that nuance. The discreteness of the model with respect to some variables that may be continuous in mathematical models may indeed be the most serious weakness of the approach described here relative to a mathematical model. But in a world in which parties make either stingy or generous offers, but with little in between, a model with ten offer level choices seems likely to be able to capture, at least approximately, the bargaining dynamics.

Figures \ref{fig:errors_american} and \ref{fig:errors_british} offer a visualization of how the plaintiff's file decision, the defendant's answer decision, the parties' offers decisions, and to a much lesser extent their decisions whether to quit after settlement failure, directly affect the variables of ultimate interest: accuracy and expenditures. Inaccuracy is defined in two different (though generally highly correlated) ways: as false negatives or false positives. A false negative inaccuracy is measured from the perspective of the meritorious plaintiff. Every dollar by which the plaintiff's net recovery \textemdash that is, damages received minus plaintiff's litigation expenditures \textemdash falls short of what the plaintiff ideally would receive from a truly liable defendant in a hypothetical perfectly accurate and costless legal system counts as a false negative dollar. A false positive inaccuracy is measured from the perspective of the truly not liable defendant. Every dollar that such a defendant spends, whether on litigation or in damages, counts as a false positive dollar. These values can be calculated directly, because the model explicitly calculates the proportion of cases in which the defendant is truly liable or not. The mathematical modeler, as we will see later, may need to resort to much less direct measures of inaccuracy that fail to take into account actual litigation expenditures.

\begin{figure}
    \centering
    \begin{minipage}{0.48\textwidth}
        \centering
        \includegraphics[width=0.98\textwidth, scale=0.70, trim={0in 0in 0in 0in}, clip]{../Figures/Source of error costs and expenditures (American)} % first figure itself
        \caption{Errors and expenditures (American)}
		\label{fig:errors_american}
    \end{minipage}\hfill
    \begin{minipage}{0.48\textwidth}
        \centering
        \includegraphics[width=0.98\textwidth, scale=0.70, trim={0in 0in 0in 0in}, clip]{../Figures/Source of error costs and expenditures (British)} % second figure itself
        \caption{Errors and expenditures (British)}
		\label{fig:errors_british}
    \end{minipage}
\end{figure}

Figures \ref{fig:errors_american} and \ref{fig:errors_british} provide a clean visual depiction of both error costs and expenditures, which have social welfare significance independent of their effects on accuracy. Nonetheless, these diagrams may require some explanation. Every case is accounted for on the vertical axis, and cases are ordered by their type of disposition, with cases in which the plaintiff doesn't file suit and the defendant does not answer at the bottom, and cases that settle (under these parameters, only with the British rule), that result in a party quitting after settlement failure (just a few cases with the British rule, too narrow to fit a label), or that result in a trial with the plaintiff losing or winning pile on toward the top. In areas of the diagrams where the false negatives and false positives are empty, this is because those cases do not produce any false positives. For example, the vast majority of cases in which the plaintiff does not file suit are cases in which the defendant is truly not liable, so this resolution is nirvana, producing zero error costs and zero expenditures. A few cases, however, are cases in which the defendant is truly liable, resulting in false negatives, represented by the rectangle with north east lines in the bottom left of the figure. (All diagrams are designed with patterns to be presentable in black and white, but the color versions, in which this box is blue, are available in the Supplemental Materials.) The height of this rectangle represents the proportion of cases of this type, and the width represents the size of the false negatives, large but not as great as would be if the parties spent money on litigating first. 

The cases in which the defendant does not answer produce similarly low error rates, but not as low, because the plaintiff bears the filing costs. When cases settle, there are some false positives and false negatives, as well as a moderate amount of expenditures from both parties' initial expenditures. Finally, when the cases go to trial, expenditures are at their maximum, and because the court is not perfect, some errors result: false positives when a nonmeritorious plaintiff wins and false negatives when a meritorious plaintiff loses. Expenditures may cause some inaccuracy even in cases in which the correct party wins. For example, when the plaintiff loses in a case in which the defendant truly is not liable, under the American rule, the defendant suffers a false positive as a result of litigation expenditures. Similarly, when the plaintiff wins in a case in which the defendant truly is liable under the British rule, no false negatives occur because the plaintiff receives full compensation, but the defendant suffers false positives from paying even more than damages in total.

In short, the area of the rectangles represents the total size of the false negatives, false positives, and total expenditures. Thus, one can at a quick glance appreciate the social welfare effects of fee shifting, holding all other parameter values constant. In this example and in most others, the effects are not all that great. Still, the British rule may have a slight edge. The British rule fails to increase the chance that the plaintiff will not file or that the defendant will not answer, a change that at least at the margins would be beneficial here given the blankness of these regions, but it is no worse than the American rule. The key difference is that the slightly greater party generosity in settlement under the British rule (visible in Figures \ref{fig:offers_american} and \ref{fig:offers_british}) leads to more settlement. It is not obvious that this should be the result. While a party that expects to lose more likely than not should be more generous, a party that expects to win should be less generous. (The analysis for now ignores that parties may spend more when the British rule applies, an issue to which we will return.) The increased settlement naturally decreases expenditures in the British rule, and it also modestly decreases false negatives and false positives. This initial face-off might qualify as a narrow win for the British, but, like the Battle of Bunker Hill, it might yet give some encouragement to the American side. 

\subsection{Costs and fee-shifting multipliers}

To develop a more complete picture of how fee-shifting intensity may affect case disposition and thus accuracy and expenditures, we can vary the costs level and introduce a wider range of fee-shifting levels. Recall that in the baseline $c_{file}=c_{answer}=c_{trial}=0.15$, so if a case goes to trial, each party spends 0.3, or 30\% of the amount at stake. Yet there may be some cases in which costs, broadly conceived, may be much lower relative to stakes and others in which costs are higher. Thus, in Figure \ref{baseline}, the baseline costs levels are multiplied by each element of $\{\frac{1}{4}, \frac{1}{2}, 1, 2, 4\}$, and five fee shifting multipliers, ranging from 0 (representing the American rule) to 2 (where the loser pays twice the winner's fees) are included. Although multiple fee shifting certainly would be unusual, Polinsky and Rubinfeld (1996) \cite{polinskyrubinfeld1996} recognize that it is unclear what the optimal multiple is. After all, without a multiple, a party with a 75\% chance of winning would expect on average to bear half the per-party costs (one quarter of the parties' combined costs). Because costs are exogenous in the model, the fee shifting multiple can be viewed more generally as a range of penalties that the loser must pay to the winner.

\begin{figure}[h!]
\centering
\includegraphics[scale=0.50, trim={0in 0in 0in 0in}, clip]{../Figures/Disposition Baseline.pdf}
\caption{Dispositions with baseline parameters, varying costs and fee shifting multipliers}
\label{fig:disposition_baseline}
\end{figure}

The middle row of Figure \ref{baseline} confirms, interpolates, and extrapolates what we determined above. The stacked bars represent the disposition of cases, with each group of stacked bars reflecting the results of one simulation and different dispositions coded with different patterns. The amount of fee shifting has relatively minimal effects on the number of cases that the plaintiff does not bring or the defendant does not answer. But fee shifting does affect settlement, increasing it at least up to a multiplier of 1. For multipliers above this level, some of the settled cases do not settle and instead become cases in which the plaintiff or the defendant gives up before trial. There is less pressure to settle if there is a good chance that the other party will quit rather than fight. Finally, the rate of trial is lower, the greater the amount of fee-shifting. 

Dari-Mattiacci and Saraceno (2020) note that the "common wisdom" is that the English rule discourages settlement. This observation dates to early papers on fee-shifting, such as Shavell (1982). The intuition is that the cases that tend to go to trial will be those in which the parties are mutually optimistic, and if the parties are sufficiently mutually optimistic, each will expect to benefit from fee-shifting, so fee-shifting makes settlement less likely. A problem with this logic is that a rational Bayesian should recognize that if trial occurs, that will be the result of mutual optimism. Dari-Mattiacci and Saraceno counter the common wisdom, in a model in which damages are uncertain, by deriving a result "that the rate of litigation [meaning trial] is independent of the fee-shifting rule." (p. 14) The data here further undermines the common wisdom by demonstrating that, at least under some parameters, settlement will become more common and trial will become less common with greater amounts of fee-shifting. This conclusion reflects not only settlement dynamics but also decisions whether to file and answer, which Dari-Mattiacci and Saraceno consider informally (p. 13). 

The result that trial levels fall with fee-shifting survives an examination of the other rows of the chart, which consider other costs levels. Moreover, the effect is even more stark if one reads the chart by moving not just to the right to observe the effect of a higher level of fee shifting, but simultaneously also up to observe the effect of a higher level of costs. Making a comparison in the style of a chess bishop may be justified because when fee shifting is greater, parties may spend more on litigation. Katz (1987) \cite{katz}, for example, estimates that changing from the American rule to the British rule might increase per case expenditures in tried cases by as much as 125\%. Thus, one might assess the effect of such a change by comparing the results associated with a 0 fee shifting multiplier and a costs multiplier of 1 with those for multipliers of 1 and 2, respectively. If fee shifting in fact makes litigation more expensive, the model here suggests that it will also make trial considerably less common.

When comparing across or diagonally, even more apparent than the change in the trial rate is the change in the rates at which suits are filed and contested.  With the lowest cost cases, suit is always filed and contested, and no one quits after settlement failure. In these cases, higher fee shifting multiples generally lead to more settlement, although there is little difference among the middle range of multipliers. With the highest costs cases, where the combined cost of trial exceeds the stakes, cases are rarely brought, and those brought are rarely defended. The remainder are mostly settled to avoid the trial costs, and if not settled, abandoned, with very few cases going to trial. One should thus expect very few contested cases to have such high costs relative to stakes. Such an absence may not mean that the litigation system always succeeds at keeping costs to a reasonable percentage of stakes, but instead that such cases are rarely litigated. This need not necessarily signify a failure of the legal system; note that about half of the uncontested cases are nonsuits and half are defaults, and many of these will end in the correct outcome with minimal if any litigation expenditures. Indeed, one might tentatively venture that error costs and litigation costs are paradoxically lower when the cost of the legal system is higher. 

On these parameters, though, that inference would be wrong. Figure \ref{fig:accexp_baseline} illustrates false negative and false positive inaccuracy, as well as average expenditures on litigation, including in the denominator all potential cases, not just those actively litigated. As costs rise, all of these lines generally rise with them. With the highest costs, false negatives are very high relative to false positives, especially with fee shifting. The timing of litigation affords the defendant an informational advantage; there are some suits that the defendant would drop that are never brought by the plaintiff. Enough of the cases are still litigated that expenditures amount to around 35\% of the stakes with the ordinary British rule. Still, this might be less than one would expect, only around three times greater than the level of total expenditures that occurs when costs are $\frac{1}{16}$ as large. The parties' greater hesitance to fight partially but incompletely offsets the increased relative cost of litigation.

\begin{figure}[h!]
\centering
\includegraphics[scale=0.50, trim={0in 0in 0in 0in}, clip]{../Figures/Accuracy and Expenditures Baseline.pdf}
\caption{Inaccuracy and expenditures with baseline parameters}
\label{fig:accexp_baseline}
\end{figure}

Within each graph within Figure \ref{fig:accexp_baseline}, the error costs and expenditure lines are relatively, though not perfectly, flat. This may seem at odds with another observation of Dari-Mattiacci and Saraceno, that fee-shifting better promotes accuracy in lower-cost litigation systems, perhaps explaining the pattern of different rules on opposite sides of the pond. The level of costs affects the level of errors, but increasing cost does not much weaken the case for fee-shifting. A close observation provides at most limited support for their claim. Error costs fall ever so slightly at the baseline costs level. (Total expenditures, a variable that they do not consider, also fall slightly.) 

\subsection{Risk aversion}

A computational model also can easily consider the effects of risk aversion. Utility with risk aversion is modeled as a function of wealth, $U=-e^{-\alpha W}$, and the values of $\alpha$ used to model risk aversion in these simulations are 1, 2, and 4. Recall that each party's initial wealth is set to 10.0, and stakes are 1.0. The shape of the risk aversion curves, with the x axis depicting wealth on the interval $[9, 11]$,  are displayed in Figure \ref{fig:riskaversion}. The y-axis utility values are omitted, because the utility values can be (and are) linearly transformed without affecting the equilibrium. 

\begin{figure}[h!]
\centering
\includegraphics[scale=0.40, trim={0in 0in 0in 0in}, clip]{../Figures/risk aversion.pdf}
\caption{Levels of risk aversion}
\label{fig:riskaversion}
\end{figure}

Figure \ref{fig:disposition_riskaversion} confirms that trial becomes considerably more rare when the parties are risk averse, especially if risk aversion is pronounced. Anticipating the increased likelihood of achieving a settlement, parties are more likely to contest claims. In the middle row, for example, as risk aversion becomes greater, the proportion of cases in which the plaintiff does not file or the defendant does not answer falls. The effect is greater when fee shifting is absent. With moderate or high risk aversion and no fee shifting, litigation is always contested at the baseline costs level, and a very high percentage of cases settle. Introduction of fee shifting, however, greatly reduces the chance of settlement success. Risk aversion exacerbates the downside risk of fee shifting, and when settlement fails, one party or the other frequently quits rather than face trial. Still, the relationships are quite complex. In the rightmost panel of the middle row, for example, the proportion of cases resulting in settlement initially falls but then rises with increased settlement. When both parties are risk averse, each player seeks to avoid trial but also recognizes that the other player is also risk averse, making trial unlikely. At the highest costs levels, however, risk aversion makes trial and even settlement very unlikely. With high risk aversion and a costs multiplier of 4, fee shifting means that the merits cease to differentiate outcomes, as either the plaintiff never files or the defendant never answers.

\begin{figure}[h!]
\centering
\includegraphics[scale=0.50, trim={0in 0in 0in 0in}, clip]{../Figures/Disposition Varying Risk Aversion.pdf}
\caption{Dispositions varying risk aversion}
\label{fig:disposition_riskaversion}
\end{figure}

Sufficiently high risk aversion makes the litigation game similar to the famous game theory game of chicken (or, for biologists, hawk-dove). In contrast to the classic game of chicken, however, the parties are not identically situated, and which party has the more credible threat to go to trial may vary based on their signals, the costs levels, and the rules. Given these complex dynamics, it is not surprising that the effects on inaccuracy and expenditures are more complex as well, especially with high levels of risk aversion, high levels of costs, and high degrees of fee shifting. The bottom three rows of Figure \ref{fig:accexp_riskaversion} suggest that mild risk aversion has relatively small effects. Overall, there is less inaccuracy and there are lower expenditures with mild risk aversion than with risk neutrality, though of course the false negative inaccuracy and false positive inaccuracy may have a greater effect on welfare. With moderate risk aversion and cost multipliers of 0.5 or 1, the benefits of fee shifting are accentuated. But with the highest levels of risk aversion, the picture becomes more complex at those costs multipliers, with moderate fee shifting multipliers performing considerably better than higher multipliers. At the highest costs level, the two inaccuracy measures sharply diverge in some places, as the party too cowardly to contest litigation suffers.  All in all, the data suggests that with mutual mild risk aversion, ordinary fee shifting will generally produce benefits, but the dynamics are more complex and predictions more uncertain with high ratios of costs to stakes.

\begin{figure}[h!]
\centering
\includegraphics[scale=0.50, trim={0in 0in 0in 0in}, clip]{../Figures/Accuracy and Expenditures Varying Risk Aversion.pdf}
\caption{Inaccuracy and expenditures with risk aversion}
\label{fig:accexp_riskaversion}
\end{figure}

\section{Robustness} \label{Robustness}

This article's methodology has made it possible to generate exact equilibria for a wide variety of game permutations, 1,250 in all. But this leaves two questions about robustness. First, are the results dependent on the happenstance of stumbling upon some equilibria rather than others? Second, if multiple equilibria do exist and the parties play different equilibria, is that likely to significantly affect the results? And third, are the results dependent on the variables left fixed across simulations, namely the number of liability (or damages) strength values, the number of signals, and the number of offers? We address these issues in turn.

\subsection{Multiple equilibria}

The von Stengel, van den Elzen, and Talman (2002) algorithm is guaranteed to furnish a single exact equilibrium given any arbitrary initialization of the parties' information sets to fully mixed strategies. It does not provide a means of generating all of the perfect Nash equilibria, in contrast to the Lemke-Howson algorithm, which generates all Nash equilibria but is many orders of magnitude too slow for this problem. This raises the question whether the equilibria identified here are representative of the broader class of equilibria for each game type. We can address this by randomizing the parties' initial information sets to determine whether that generates multiple equilibria and, if so, how much variation there is among these multiple equilibria.

We performed this for the baseline simulation, both with and without fee shifting, by attempting to generate up to 50 equilibria for each. This produced 29 equilibria for the American rule, because 21 of the equilibria were repeats, and 9 for the British rule. We cannot from this exercise determine how many equilibria exist. In principle, these equilibria could be the entire populations or they could be a small sample. Figures \ref{fig:baseline_multieq_american} and \ref{fig:baseline_multieq_british} thus reports the coefficient of variation (the standard deviation divided by the mean) for various outcome variables among the American and British rule equilibria, respectively. The results are relatively reassuring. There is some variation with the American rule, particularly in the answer and offer decisions, but this results in only very small variation for our primary outcome variables of interest, error costs and expenditures. (The "Settles" variable is omitted; there were no settlements in all but one equilibria, and in that equilibria, 10.7\% of cases settled.) The results for the British rule are even more reassuring. The only outcome variable with a nonzero coefficient of variation was the plaintiff offer, at 0.03; this variation was insufficient to actually change whether any cases settled. 

\begin{figure}
    \centering
    \begin{minipage}{0.48\textwidth}
        \centering
        \includegraphics[scale=0.5, trim={0in 0in 0in 0in}, clip]{../Figures/Baseline multiple equilibria (American).pdf}
		\caption{Coefficient of variation for 29 American rule equilibria}
\label{fig:baseline_multieq_american}
    \end{minipage}\hfill
    \begin{minipage}{0.48\textwidth}
        \centering
		\includegraphics[scale=0.5, trim={0in 0in 0in 0in}, clip]{../Figures/Baseline multiple equilibria (British).pdf}
		\caption{Coefficient of variation for 9 British rule equilibria}
		\label{fig:baseline_multieq_british}
    \end{minipage}
\end{figure}

\subsection{Off-equilibrium play}

Chicken highlights a central concern about multiple equilibria, that if they exist, the parties might not coordinate on any one equilibrium. The only pure Nash strategies in chicken are those in which one party swerves and the other does not. Either way, no one dies. But if each player believes that they are coordinating on the equilibrium in which the other swerves, then both die. This outcome is also possible, of course, if they coordinate on the mixed strategy, in which each has a high probability of swerving, but is far more likely if there is a potential failure to coordinate on pure strategies. In the context of litigation, if the plaintiff believes that the equilibrium is one in which the plaintiff drives a hard bargain and the defendant thinks that the equilibrium is one in which the defendant drives a hard bargain, then they may end up with a nonequilibrium settlement failure that would not occur if both parties were coordinated.

Given the low coefficient of variation for the baseline scenarios, the noncoordination problem appears insignificant, but that does not rule out the possibility that off-equilibrium play could be considerably different from equilibrium play in other scenarios.  It was infeasible to generate a large number of equilibria for each of the 1,250 scenarios to assess the implications of coordination failure, given available computational resources. As a substitute, we generated up to 50 equilibria per scenario on a smaller game tree size, where $n_{LS}=n_{LS}^P=n_{LS}^D=n_{\mathcal{o}}=5$. For each equilibrium attempt, we initialized the mixed strategies at random in a way that allowed for the possibility that one action might have a much higher probability than others, in the hope of creating highly diverse starting points for the algorithm. For efficiency, we then attempted to find an equilibrium using floating point arithmetic. As noted above, that is not guaranteed to succeed because of numerical stability issues, so whenever it failed, we switched to seeking an equilibrium using exact arithmetic. We then collected all the unique equilibria produced. 

We then compared outcomes in a correlated equilibrium with outcomes in an average equilibrium. A correlated equilibrium, developed by Aumann (1974), is one in which each party, observing some signal of which equilibrium to play, has the incentive to follow that signal on the assumption that the other party will as well. Any set of Nash equilibria can be combined to create a correlated equilibrium, because if one knows that one's opponent is playing a particular Nash equilibrium, by definition one will want to play the strategy corresponding to that equilibrium as well. The average equilibrium is calculated simply by averaging the parties' mixed strategies at each information set. For example, if in half of equilibria, plaintiff files 100\% of the time conditional on a signal and in the other half, plaintiff files 50\% of the time conditional on that signal, then in the average equilibrium, plaintiff files 75\% of the time. The average equilibrium will thus have support over at least as many pure strategies as any of the original equilibria strategies. 

For each of the 1,250 parameter sets, we created a correlated equilibrium strategy and an average equilibrium strategy and computed the probability or average value of game outcomes for each. We then calculated how correlated the variables were for a number of variables of interest. The result is illustrated in Figure \ref{fig:correlations_corrvsave}, which shows the correlation coefficient across the 1,250 values for each outcome variable. This suggests that lack of ability to coordinate on any single strategy does not dramatically change the story.  

\begin{figure}[h!]
\centering
\includegraphics[scale=0.5, trim={0in 0in 0in 0in}, clip]{../Figures/Correlation between correlated and average equilibrium.pdf}
\caption{Correlations in outcome variables between correlated and average equilibria}
\label{fig:correlations_corrvsave}
\end{figure}

\subsection{Game tree size}

The equilibria on the smaller game tree also can be compared directly to the equilibria on the original game tree. This provides a test of whether results are highly sensitive to the granularity of the strategies. The diagrams illustrating equilibrium results for both the smaller and original game tree are available in the Supplemental Materials. We will offer here a partial visual comparison. Figures \ref{fig:treesize_panel1}-\ref{fig:treesize_panel4} compare the large and small trees, with dispositions under risk neutrality on the left and dispositions under mild risk aversion on the right. The results highlight that tree size can matter for some parameters. Particularly for the 0.5 costs multiple, settlements are more common on the small tree, for example. This may be because it is easier to coordinate on a settlement when there are only a handful of settlement choices than when there are a larger number. This might suggest that settlement might be slightly more common if liability strengths, signals, and offers were doubled yet again from our baseline parameters, with other resulting adjustments such as a greater chance of suit. Yet the broad general patterns of how dispositions change with costs and with greater intensity of fee shifting remain. 

\begin{figure}
    \centering
    \begin{minipage}{0.24\textwidth}
        \centering
        \includegraphics[scale=0.15, trim={0in 0in 0in 0in}, clip]{../Figures/Disposition Baseline.pdf}
		\caption{Risk neutral, large tree}
		\label{fig:treesize_panel1}
    \end{minipage}\hfill
    \begin{minipage}{0.24\textwidth}
        \centering
		\includegraphics[scale=0.15, trim={0in 0in 0in 0in}, clip]{../Figures/Disposition Baseline (Small Tree).pdf}
		\caption{Risk neutral, small tree}
		\label{fig:treesize_panel2}
    \end{minipage}
   \begin{minipage}{0.24\textwidth}
        \centering
        \includegraphics[scale=0.15, trim={0in 0in 0in 0in}, clip]{../Figures/Disposition (Risk Averse) Baseline.pdf}
		\caption{Risk averse, large tree}
		\label{fig:treesize_panel3}
    \end{minipage}\hfill
    \begin{minipage}{0.24\textwidth}
        \centering
		\includegraphics[scale=0.15, trim={0in 0in 0in 0in}, clip]{../Figures/Disposition (Risk Averse) Baseline (Small Tree).pdf}
		\caption{Risk averse, small tree}
		\label{fig:treesize_panel4}
    \end{minipage}
\end{figure}

\section{Analysis and Conclusion}

The model has provided some surprises, while confirming some standard findings of the literature. Even when two risk-neutral parties have relatively strong information, a great deal of strategic behavior can occur and thwart settlement in a perfect Bayesian Nash equilibrium. Settlement offers at equilibrium rarely rise linearly with parties' estimates of the stakes, but change abruptly from stinginess to generosity. Fee shifting has relatively little effect on the rate at which litigation is contested, at least unless the parties are significantly risk averse. When litigation becomes more expensive, fewer cases are contested, but the greater per-case expense and gamesmanship in file/answer and quitting decisions generally worsens social welfare outcomes. At very high levels of costs, litigation becomes akin to a game of chicken. One or both parties may be bluffing by filing or answering, settlement becomes rare, and inaccuracy is high. 

Perhaps the most decisive conclusion that can be drawn from the constellation of parameter values explored in this article's preliminary analysis of the model results is that litigation may play out quite differently with different costs levels. An advantage of using a common model in different scenarios is that we can be more comfortable attributing the difference to the changes in parameters than we would be if comparing the implications of pairs of models making multiple different assumptions. The analysis illustrates more clearly than is possible under other models the complex interactions among file/answer decisions, settlement offers, and quit options, as well as how these change depending on case stakes, the degree of fee shifting, risk aversion, and a number of other variables. The comparisons suggest that the long-term goal must be either to develop different rules for different case types or to develop alternative litigation rules that tend to maximize social welfare under a wide range of parameters. 

With respect to fee-shifting, the project sheds light on how the presence or degree of fee-shifting affects the incidence of trial and social welfare measures. The most sophisticated model addressing these questions to date is Dari-Mattiacci and Saraceno (2020), who incorporate two-sided asymmetric information and fee-shifting. They conclude that fee-shifting does not affect the settlement rate and that fee-shifting tends to produce greater accuracy when litigation costs are relatively low. As detailed above, the simulations here seem largely inconsistent with these conclusions. 

Dari-Mattiacci and Saraceno's model includes a number of restrictive assumptions. A single variable serves two entirely separate functions in the model, representing both the true merits of the case and the degree of information asymmetry, thus preventing these considerations from being considered separately. That variable is then constrained to an intermediate range of values. The parties share knowledge of that variable and thus have information only about how factors independent of the truth may influence the judge. The imposition of fee shifting depends on whether the party that wins itself presented strong evidence, ignoring the other party's case.  The parties are arguing not over liability but over how to divide a disputed asset. The plaintiff always files, and the defendant always contests. The measures of inaccuracy take into account fee shifting but ignore the costs that can contribute to false positives and false negatives, and the measure compares the expectation of outcomes with the true merits, an approach that works only with risk neutrality. 

These observations are not intended as criticism. These sacrifices were needed and worthwhile to ensure tractability in a mathematical model. By translating litigation games into extensive form and applying a game theoretic algorithm, this article has allowed for a far wider range of variables to be considered and has avoided some of these sacrifices. Still, there may be situations in which the Dari-Mattiacci and Saraceno model may provide the better analysis, particularly where both parties know the truth of the matter, but neither party knows the evidence that the other may be able to assemble to convince the judge to award it more of a disputed asset.

At least three forms of skepticism about the methodology here are warranted. First, litigants may not be rational Bayesians who play equilibrium strategies. But, as the literature generally presumes, models assuming rationality may provide a first guess at how litigants behave, and economic forces may gradually push actors in any given legal context toward equilibrium strategies.  Moreover, though more complicated to model, two-sided asymmetric information seems like a more accurate description of most litigation than one-sided information, and a model that recognizes the possibility that litigants may not contest litigation or may quit after settlement failure illustrates dynamics that some models assume away. 

Second, models' usefulness may depend on parameter values, and the parameters here are not calibrated based on data, either statistics from actual litigation or on microfoundations of individual behavior. Indeed, the heterogeneity of results that obtain from this model's admittedly arbitrary parameter choices highlight both the futility of assuming that any calibration will fit all litigation but also that calibration may be useful for particular types of cases. Risk aversion, for example, has a profound effect on the results, and even casual observation suggests that the high trial rates observed with risk neutrality are less plausible than the rates observed with mild risk aversion. The model also could be modified to incorporate behavioral considerations like regret aversion, which, experimental evidence in Guthrie (1999) \cite{guthrie} suggests, affects litigant behavior.

Third, although the model offers more bells and whistles than its mathematical predecessors (and full description of the results presented in the Supplemental Materials will require future papers), many chimes are still missing, including learning and bargaining over time, as well as the role of lawyers and agency costs. Indeed, this is but a first step toward building models that can incorporate even more complex features of the litigation game. Because the game tree size may increase exponentially with additional features, other computational game theory approaches that seek approximate equilibria may be needed to make larger game models solvable. 

\printbibliography
\end{document}
